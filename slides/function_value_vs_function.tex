\begin{frame}[fragile]
    \frametitle{Function evalution vs. Function representation}
    \colemph{Question}: What about plotting $\sin(u_h)$? And $\nabla u_h$
    and $| \nabla u_h |$? \\
    \colemph{Experiment:} Try it out! Use
    \vspace{-1em}
    \begin{python}
sqrt(grad(u)**2)
    \end{python}
    for $|\nabla u |$.
    What happens if you plot these function? Have a closer
    look at the terminal output. Anything suspicious?
%    \colempph{Experiment:} Try the same for Poisson example $1$

    \vspace{1em}
\colemph{Question:} What happened now? Why is there a \\
\texttt{> Object cannot be plotted directly, projecting to piecewise
linears.}\\
\colemph{Answer:}
\begin{itemize}

%From a implementation perspective, two completely
%different things happen behind the scene:
%\begin{equation*}
%    \underbrace{\sin}_{\text{\small
%    built-in}}(\underbrace{u_h(x)}_{\text{Function value}})
%    \qquad \text{vs.} \sin \circ u_h
%\end{equation*}
%\\
    \item $\sin(u_h(x))$ is the evaluation of the built-in function $\sin$ at a
        \emph{given} value $u_h(x)$, which in turn results from a
        FEM function evalution. \\
    \item  $\sin \circ u_h$ is a composition of the built-in function $\sin$ and a
        FEM function $u_h$. The composition is a symbolic \texttt{UFL} (Unified
        Form Language) expression.
\end{itemize}
\end{frame}
