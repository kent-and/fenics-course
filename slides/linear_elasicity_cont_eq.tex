\begin{frame}
\frametitle{Linear elasticity: continuous equations}
\begin{itemize}
\item  $\Omega_0 \in \mathbb{R}^3$ that is being deformed under a load
\item  $\Omega$ is the deformed $\Omega_0$
\item  Let $x\in \Omega$ correspond to $X\in\Omega_0$
\item  Then the deformation is $ u = x - X$
\item  The stress tensor $\sigma$ is a symmetric $3\times 3$ tensor which is a function of $u$
\item Hooke's law states:
\[
\sigma = 2 \mu \epsilon(u) + \lambda \operatorname{tr}(\epsilon(u)) \delta
\]
\item  In equilibrium (i.e. no acceleration terms), Newton's second law states:
\begin{eqnarray*}
-\operatorname{div} \sigma &=& f, \quad\mbox{in}\ \Omega   \\
\sigma \cdot n &=& g, \quad\mbox{on}\ \partial \Omega
\end{eqnarray*}
\item $f$ and $g$ are body and surface forces
\item $n$ is the outward normal vector
\end{itemize}

\end{frame}
