\begin{frame}[fragile]
  \frametitle{Automated goal-oriented adaptivity -- A complete example}
  \vspace{-2.0em}
\begin{python}
from fenics import *

# Create mesh and define function space
mesh = UnitSquareMesh(8, 8)
V = FunctionSpace(mesh, "Lagrange", 1)

# Define boundary condition
u0 = Function(V)
bc = DirichletBC(V, u0, "x[0] < DOLFIN_EPS || x[0] > 1.0 - DOLFIN_EPS")
\end{python}
\end{frame}


\begin{frame}[fragile]
  \frametitle{Automated goal-oriented adaptivity -- A complete example}
Define variational problem:
\vspace{-1.0em}
\begin{python}
u = TrialFunction(V)
v = TestFunction(V)
f = Expression("10*exp(-(pow(x[0] - 0.5, 2) + pow(x[1] - 0.5, 2)) / 0.02)",
               degree=1)
g = Expression("sin(5*x[0])", degree=1)
a = inner(grad(u), grad(v))*dx
L = f*v*dx + g*v*ds
\end{python}
\end{frame}

%\begin{frame}[fragile]
%  \frametitle{Automated goal-oriented adaptivity -- A complete example}
%\vspace{-1.0em}
%\begin{python}
%u = TrialFunction(V)
%v = TestFunction(V)
%f = Expression("10*exp(-(pow(x[0] - 0.5, 2) + pow(x[1] - 0.5, 2)) / 0.02)",
%               degree=1)
%g = Expression("sin(5*x[0])", degree=1)
%a = inner(grad(u), grad(v))*dx
%L = f*v*dx + g*v*ds
%\end{python}
%\end{frame}

\begin{frame}[fragile]
  \frametitle{Automated goal-oriented adaptivity -- A complete example}
Define function for the solution:
\vspace{-1.0em}
\begin{python}
u = Function(V)
\end{python}
\bigskip
Define goal functional (quantity of interest) and tolerance:
\vspace{-1.0em}
\begin{python}
M = u*dx
tol = 1.e-5
\end{python}
\end{frame}

\begin{frame}[fragile]
  \frametitle{Automated goal-oriented adaptivity -- A complete example}
Solve equation a = L with respect to u and the given boundary
conditions, such that the estimated error (measured in M) is less
than tol
\vspace{-1.0em}
\begin{python}
solver_parameters = {"error_control":
                     {"dual_variational_solver":
                      {"linear_solver": "cg"}}}
solve(a == L, u, bc, tol=tol, M=M, solver_parameters=solver_parameters)
\end{python}
\end{frame}

\begin{frame}[shrink=5,fragile]
  \frametitle{Automated goal-oriented adaptivity -- A complete example}
Alternative, more verbose version (+ illustrating how to set parameters)
\vspace{-1.0em}
\begin{python}
problem = LinearVariationalProblem(a, L, u, bc)
solver = AdaptiveLinearVariationalSolver(problem)
solver.parameters["error_control"]["dual_variational_solver"]["linear_solver"] = "cg"
solver.solve(tol, M)
\end{python}
\bigskip
Extract solutions on coarsest and finest mesh:
\bigskip
\vspace{-1.5em}
\begin{python}
plot(u.root_node(), title="Solution on initial mesh")
plot(u.leaf_node(), title="Solution on final mesh")
interactive()
\end{python}
\end{frame}

%\begin{frame}
%  \frametitle{FEniCS coding exercise}
%  Implement the problem described in the walk-through example!
%  Afterwards modify it in such a way that you solve the same problem
%  as in the coding exercise for the residual error estimator.
%  As goal functional compute
%  \begin{equation*}
%    \goal(u) = \int_{\partial \Gamma} u\dS
%  \end{equation*}
%  where $\Gamma$ is given by
%  \begin{equation*}
%    \Gamma = [0.5,1] \times \{1\} \cup \{1\} \times [0.5,1]
%  \end{equation*}
%  Choose different TOL starting from TOL=$0.01$. How do the generated
%  meshes compare to the generated meshes in the first exercise?
%\end{frame}
