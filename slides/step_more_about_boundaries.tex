\begin{frame}[fragile, shrink=5]
  \frametitle{Step by step: more about defining domains}
  For a Dirichlet boundary condition, a simple domain can be defined
  by a string
  \vspace{-1em}
  \begin{python}
"on_boundary" # The entire boundary
  \end{python}

  Alternatively, domains can be defined by subclassing \emp{SubDomain}
  \vspace{-1em}
  \begin{python}
class Boundary(SubDomain):
    def inside(self, x, on_boundary):
        return on_boundary
  \end{python}

  You may want to experiment with the definition of the boundary:
  \vspace{-1em}
  \begin{python}
"near(x[0], 0.0)" # x_0 = 0
"near(x[0], 0.0) || near(x[1], 1.0)"
  \end{python}

  There are many more possibilities, see
  \vspace{-1em}
  \begin{python}
help(SubDomain)
help(DirichletBC)
  \end{python}

\end{frame}
