\begin{frame}[fragile]
  \frametitle{Step by step: solving (nonlinear) variational problems}

  Once a variational problem has been defined, it may be solved
  by calling the \emp{solve} function (as for linear problems):
\vspace{-1em}
  \begin{python}
solve(F == 0, w, bc)
  \end{python}

Or more verbosely
\vspace{-1em}
  \begin{python}
dF = derivative(F, w)
pde = NonlinearVariationalProblem(F, w, bc, dF)
solver = NonlinearVariationalSolver(pde)
solver.solve()
  \end{python}

Extracting the subfunctions (as DOLFIN functions)
\vspace{-1em}
  \begin{python}
(u, p) = w.split(deepcopy=True)
  \end{python}



\end{frame}
