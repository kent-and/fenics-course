\begin{frame}
\frametitle{Linear elasticity: continuous equations, cont'd}
\begin{itemize}
\item Hooke's law states:
\[
\sigma = 2 \mu \epsilon(u) + \lambda \operatorname{tr}(\epsilon(u)) \delta
\]
\item $\epsilon(u)$ is the strain tensor or the symmetric gradient:
\[
\epsilon(u) = \frac{1}{2} (\nabla u + (\nabla u)^T)
\]
\item $\mu$ and $\lambda$ are the Lame constants
\item $\operatorname{tr}$ is the trace operator (the sum of the diagonal matrix
entries), $u$ is the displacement, and
\[
\delta = \left[ \begin{array}{ccc} 1 & 0 & 0 \\ 0 & 1 & 0 \\ 0 & 0 & 1 \end{array} \right]
\]

From Newton's second law and Hooke's law we arrive directly at the equation of linear elasticity:
\begin{align}
\label{el:eq}
-2 \mu (\nabla \cdot \epsilon (u)) - \lambda \nabla (\nabla \cdot u) = f
\end{align}
\end{itemize}
\end{frame}

\begin{frame}
\frametitle{Material parameters for linear isotropic elasticity}

Lam\'e parameters:
\begin{align}
\mu &= \frac{E}{2\cdot(1 + \nu)} \\
\lambda &= \frac{E\nu}{(1 + \nu)(1 - 2\nu)}
\end{align}

\begin{itemize}
\item
  The first Lam\'e parameter $\mu$ relates shear stress to shear strain ($\sigma_{xy} = 2\mu\epsilon_{xy}$)
\item
  The second Lam\'e parameter $\lambda$ relates hydrostatic pressure to volumetric strain
\item
  The Young's modulus $E$ relates axial stress to axial strain ($\sigma_{xx} = E\epsilon_{xx}$)
\item
  The Poisson ratio $\nu$ relates (negative) transversal compression to axial stretch ($\nu = -\Delta y / \Delta x$)
\end{itemize}

Note: $\nu = 0.5$ and $\lambda = \infty$ for an incompressible material.

\end{frame}
