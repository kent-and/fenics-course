\begin{frame}
\frametitle{Boundary conditions cont'd}
We may derive boundary conditions for 
$\phi$ in two ways: 
\begin{itemize}
\item From the scheme:  
\[
u^{n+1} =  u^*- \frac{\Delta t}{\rho} \nabla \phi 
\]
Since $u^{n+1}$ and $u^*$ have the same boundary conditions, we obtain homogenous
Neumann conditions for $\phi$, i.e., 
\[
\nabla \phi \cdot n  = \frac{\rho}{\Delta t}(u^{n+1} -  u^*) \cdot n = 0   
\]
\item From the Navier-Stokes equations we have that :  
\[
 \nabla p^n      =  - \rho (\frac{\partial u^n}{\partial t} + u^n \cdot \nabla u^n)
  + \mu \nabla^2 u^n + f 
\]
Since $\phi = p^{n+1} - p^n$ and $p^{n+1} \not = p^n$
we obtain a non-homogenous condition. 
\end{itemize}
In conclusion: We arrive at two different conditions which both 
seem perfectly reasonable. The difference is first order.  
\end{frame}
