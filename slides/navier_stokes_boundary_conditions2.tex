\begin{frame}
  \frametitle{Boundary conditions}

  \vspace{1em}
  We consider boundary conditions of the type
  \begin{alignat*}{2}
    u &= g_{_\mathrm{D}} && \quad \text{ on } \Gamma_{_\mathrm{D}} \times (0, T] \\
    \sigma \cdot n &= t_{_\mathrm{N}} = - \bar{p} n  && \quad \text{ on } \Gamma_{_\mathrm{N}} \times (0, T]
  \end{alignat*}

  \begin{itemize}
  \item
    Velocity boundary conditions ($u = g_D$ on $\Gamma_D$) are
    enforced strongly in the finite element spaces, i.e as Dirichlet
    boundary conditions.
  \item
    Traction boundary conditions ($\sigma \cdot n = t_{_\mathrm{N}}$)
    are enforced weakly in the variational formulation.
  \end{itemize}

  In the splitting scheme, auxilliary boundary conditions are
  required for the pressure Poisson problem:
  \begin{align*}
    p &= \bar{p} \quad \text{ on } \Gamma_{_\mathrm{N}} \\
    \partial_n \dot{p} &= 0 \quad \text{ on } \Gamma_{_\mathrm{D}}
  \end{align*}
  and an auxiliary initial condition for the pressure $p^{-1/2} = p_0$.

\end{frame}

\begin{frame}
  \frametitle{Enforcing traction boundary conditions in splitting scheme}
  Note that
  \begin{equation*}
    \begin{split}
      \sigma(u^{n-1/2}, p^{n-3/2})
      &= 2 \mu \varepsilon (u^{n-1/2})  - p^{n-3/2} I - p^{n-1/2} I + p^{n-1/2} I \\
      &= \sigma(u^{n-1/2}, p^{n-1/2}) + p^{n-1/2} - p^{n-3/2}
    \end{split}
  \end{equation*}
  If we want to enforce
  \begin{equation*}
    \sigma(u^{n-1/2}, p^{n-1/2}) \cdot n = - \bar{p} n
  \end{equation*}
  and
  \begin{equation*}
    p^{n - 1/2} \cdot n = \bar{p} n
  \end{equation*}

  Then, we should say
  \begin{equation*}
    \sigma(u^{n-1/2}, p^{n-3/2}) \cdot n = - p^{n-3/2} n
  \end{equation*}
\end{frame}
