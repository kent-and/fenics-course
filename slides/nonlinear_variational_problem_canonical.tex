\begin{frame}%[shrink=10]
  \frametitle{Canonical nonlinear variational problem}
  \bigskip

  The following canonical notation is used in FEniCS for (possibly)
  nonlinear problems:

  \bigskip

  \begin{columns}[t]
    \begin{column}{0.4\textwidth}
      Find $w \in W$ such that
      \begin{equation*}
        F(w; y) = 0
      \end{equation*}
      for all $y \in \hat{W}$.
    \end{column}
    \begin{column}{0.4\textwidth}
      \vspace{-2em}
      {\small
      \begin{block}{Note}
        Here, $w$ is a function, and $y$ is a test function, and so $F$ is
        a \emph{linear form}.
      \end{block}
      }
    \end{column}
  \end{columns}

  \bigskip
  \begin{block}{For the Stokes example}
  The functions are $w = (u, p)$, $y = (v, q)$ and the form $F$ is
  \begin{equation*}
    \begin{split}
    F(w; y) =
    &\int_{\Omega} 2\nu \epsilon(u) \cdot \Grad v \dx
    - \int_{\Omega} p \Div v \dx
    - \int_{\Omega} \Div u \, q \dx \\
    &+ \int_{\partial \Omega_N} p_0 \, v \cdot n \ds
    \end{split}
  \end{equation*}
  \end{block}
\end{frame}
