\begin{frame}
\frametitle{Linear elasticity: continuous equations, cont'd}
The equation of linear elasticity: 
\begin{align}
\label{el:eq}
-2 \mu (\nabla \cdot \epsilon (u)) - \lambda \nabla (\nabla \cdot u) = 0
\end{align}
The equation is  elliptic, but there are crucial differences between this equation
and a standard elliptic equation like $-\Delta u = f$. These differences often cause
problems in a numerical setting. To explain the numerical issues we will here
consider the three operators:
\begin{enumerate}
\item $\Delta  = \nabla\cdot\nabla  = \operatorname{div}\operatorname{grad} $
\item $\nabla \cdot \epsilon = \nabla\cdot(\frac{1}{2}(\nabla + (\nabla^T))$ 
\item $\nabla \cdot \operatorname{tr} \epsilon = \nabla \nabla \cdot = \operatorname{grad} \operatorname{div}$ 
\end{enumerate}
Item 2 leads to the study of rigid motions while item 3 leads to the study of locking 
\end{frame}
