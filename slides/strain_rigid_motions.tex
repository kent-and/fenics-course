\begin{frame}
\frametitle{The strain of a rigid motion}
The strain of a rigid motion is zero. \alert{CHECK!} Consequences:  
\begin{itemize}
\item Let $r$ be a rigid motion 
\item then $\epsilon(r) = 0$ and therefore 
\begin{itemize} 
\item $\nabla \cdot \epsilon (r) = 0$ 
\item $\operatorname{tr} \epsilon(r) = 0$ 
\item $\nabla\cdot \operatorname{tr} \epsilon(r) = 0$ 
\end{itemize}
\end{itemize}

Our equation
\[
-2 \mu (\nabla \cdot \epsilon (u)) - \lambda \nabla (\nabla \cdot u) = f 
\]
is singular since if $u$ solves the equation then so does $u+r$ because 
\[ 
-2 \mu (\nabla \cdot \epsilon (r)) - \lambda \nabla (\nabla \cdot r) = 0
\]
The same applies to Neumann conditions. 
Therefore the pure Neumann problem is the hardest! 

\end{frame}
