%% MER: There is also an alternate version of Lecture 02 developed by
%% Andre, see lecture_02_static_linear_pdes_alt.tex

\documentclass{fenicscourse}

\begin{document}

\fenicslecture{Lecture 2: Static linear PDEs}
              {Hans Petter Langtangen,
               Anders Logg \\
               Marie E. Rognes,
               Andr\'e Massing \\}

%% General intro
%% -----------------------------------------------------------------------------
\begin{frame}
  \frametitle{Hello World!}

  We will solve Poisson's equation, the Hello World of scientific
  computing:
  \begin{equation*}
    \begin{split}
      - \Delta u &= f \,\,\, \quad \mbox{in } \Omega
      \\
    u &= u_0 \quad \mbox{on } \partial \Omega
    \end{split}
  \end{equation*}

  Poisson's equation arises in numerous contexts:
  \begin{itemize}
  \item
    heat conduction, electrostatics, diffusion
    of substances, twisting of elastic rods, inviscid fluid flow, water
    waves, magnetostatics
  \item
    as part of numerical splitting strategies of more complicated
    systems of PDEs, in particular the Navier--Stokes equations
  \end{itemize}

\end{frame}

\begin{frame}
  \frametitle{The FEM cookbook}

  \def\svgwidth{1.05\textwidth}
  \import{pdf/}{pdf/fem_steps.pdf_tex}

\end{frame}
 % Comment this out if FEM intro is covered in detail
\begin{frame}
  \frametitle{Solving PDEs in FEniCS}

  Solving a physical problem with FEniCS consists of the following steps:
  \begin{enumerate}
  \item
    Identify the PDE and its boundary conditions
  \item
    Reformulate the PDE problem as a variational problem
  \item
    Make a Python program where the formulas in the variational
    problem are coded, along with definitions of input data such as $f$,
    $u_0$, and a mesh for $\Omega$
  \item
    Add statements in the program for solving the variational problem,
    computing derived quantities such as $\nabla u$, and visualizing
    the results
  \end{enumerate}

\end{frame}


%% Deriving the variational formulation
%% -----------------------------------------------------------------------------
\input{slides/variational_problem_multiply}
\input{slides/variational_problem_formulation}
\input{slides/variational_problem_discrete}
\begin{frame}
  \frametitle{Canonical variational problem}

  The following canonical notation is used in FEniCS: find
  $u \in V$ such that
  \begin{equation*}
    a(u, v) = L(v)
  \end{equation*}
  for all $v \in \hat{V}$

  \bigskip

  For Poisson's equation, we have
  \begin{align*}
    a(u, v) &= \int_{\Omega} \nabla u \cdot \nabla v \dx
    \\
    L(v) &= \int_{\Omega} fv \dx
  \end{align*}

  \bigskip
  $a(u, v)$ is a \emph{bilinear form} and $L(v)$ is a \emph{linear form}

\end{frame}


%% Setting up an MMS
%% -----------------------------------------------------------------------------
\begin{frame}
  \frametitle{A test problem}

  We construct a test problem for which we can easily check the
  answer. We first define the exact solution by
  \begin{equation*}
    u(x, y) = 1 + x^2 + 2y^2
  \end{equation*}

  \bigskip

  We insert this into Poisson's equation:

  \begin{equation*}
     f = -\Delta u = -\Delta (1 + x^2 + 2y^2) = -(2 + 4) = -6
  \end{equation*}

  \bigskip

  This technique is called the \emph{method of manufactured solutions}

\end{frame}


%% FEniCS implementation: overview and each step in detail
%% -----------------------------------------------------------------------------

%% MER: There are two versions here: Alt 1 is the version from the
%% tutorial (in which on_boundary is defined via a Python
%% function). Alt 2 is the simpler version using 'on_boundary'
%% directly. Marie prefers Alt 2.

%\begin{frame}[fragile]
  \frametitle{Implementation in FEniCS}

  % This is d1_p2D.py from the FEniCS Tutorial with some minor changes:
  % - comments removed
  % - moved definition of f before u and v
  % - last lines removed

  \begin{python}
from fenics import *

mesh = UnitSquareMesh(6, 4)
V = FunctionSpace(mesh, "Lagrange", 1)
u0 = Expression("1 + x[0]*x[0] + 2*x[1]*x[1]", degree=2)

def u0_boundary(x, on_boundary):
    return on_boundary

bc = DirichletBC(V, u0, u0_boundary)

f = Constant(-6.0)
u = TrialFunction(V)
v = TestFunction(V)
a = inner(grad(u), grad(v))*dx
L = f*v*dx

u = Function(V)
solve(a == L, u, bc)
  \end{python}

% Last part of the program does not fit on page
%plot(u)
%plot(mesh)
%
%file = File("poisson.pvd")
%file << u
%
%interactive()

\end{frame}
                %% Alt 1
\begin{frame}[fragile]
  \frametitle{Implementation in FEniCS}
  \begin{python}
from fenics import *

mesh = UnitSquareMesh(8, 8)
V = FunctionSpace(mesh, "Lagrange", 1)

u0 = Expression("1 + x[0]*x[0] + 2*x[1]*x[1]", degree=2)
bc = DirichletBC(V, u0, "on_boundary")

f = Constant(-6.0)
u = TrialFunction(V)
v = TestFunction(V)
a = inner(grad(u), grad(v))*dx
L = f*v*dx

u = Function(V)
solve(a == L, u, bc)

plot(u)
interactive() # If using VTK plotting
  \end{python}

\end{frame}
 %% Alt 2

%\begin{frame}[fragile]
\frametitle{Code conventions used in these lectures}

This is Python code (write it in an Python shell or in your text editor):
\vspace{-1em}
\begin{python}
a = 1
\end{python}

This is Shell code (write what follows the \$ in a terminal):
\vspace{-1em}
\begin{python}
$ python test.py
\end{python}

\end{frame}


\begin{frame}[fragile]
  \frametitle{Step by step: the first line}

  The first line of a FEniCS program usually begins with

  \begin{python}
from fenics import *
  \end{python}

  \bigskip

  This imports key classes like \emp{UnitSquareMesh}, \emp{FunctionSpace},
  \emp{Function} and so forth, from the FEniCS user interface
  (DOLFIN)

\end{frame}

\begin{frame}[fragile]
  \frametitle{Step by step: creating a mesh}

  Next, we create a mesh of our domain $\Omega$:
  \vspace{-1em}
  \begin{python}
mesh = UnitSquareMesh(8, 8)
  \end{python}
  This defines a mesh of $8 \times 8 \times 2 = 128$ triangles of the
  unit square.

  \bigskip

  Other useful classes for creating built-in meshes include
  \emp{UnitIntervalMesh},
  \emp{UnitCubeMesh},
  \emp{UnitCircleMesh},
  \emp{UnitSphereMesh},
  \emp{RectangleMesh} and
  \emp{BoxMesh}

  \bigskip

  More complex geometries can be built using Constructive Solid
  Geometry (CSG) through the FEniCS component \emp{mshr}:
  \vspace{-1em}
  \begin{python}
from mshr import *
r = Rectangle(Point(0.5, 0.5), Point(1.5, 1.5))
c = Circle(Point(1.0, 1.0), 0.2)
g = r - c
mesh = generate_mesh(g, 10)
  \end{python}

  \normalsize

\end{frame}

\begin{frame}[fragile]
  \frametitle{Step by step: creating a function space}

  The following line creates a finite element function space relative to this mesh:
\vspace{-1em}
\begin{python}
V = FunctionSpace(mesh, "Lagrange", 1)
\end{python}

  \bigskip

  The second argument specifies the type of element, while the third
  argument is the degree of the basis functions on the element

  \bigskip

  Other types of elements include
  \emp{"Discontinuous Lagrange"},
  \emp{"Brezzi-Douglas-Marini"},
  \emp{"Raviart-Thomas"},
  \emp{"Crouzeix-Raviart"},
  \emp{"Nedelec 1st kind H(curl)"} and
  \emp{"Nedelec 2nd kind H(curl)"}

\end{frame}

\begin{frame}[fragile]
  \frametitle{Step by step: defining expressions}

  Next, we define an expression for the boundary value:
  \vspace{-1em}
  \begin{python}
u0 = Expression("1 + x[0]*x[0] + 2*x[1]*x[1]", degree=2)
  \end{python}
  The formula must be written in C++ syntax, and
  the polynomial degree must be specified.

  \bigskip

  The \emp{Expression} class is very flexible and can be used to
  create complex user-defined expressions. For more information, try
\vspace{-1em}
  \begin{python}
from fenics import *
help(Expression)
  \end{python}
  in Python or, in the shell:
  \vspace{-1em}
  \begin{python}
$ pydoc fenics.Expression
  \end{python}

\end{frame}


%\begin{frame}[fragile]
  \frametitle{Step by step: defining boundaries}

  We next define the Dirichlet boundary:
  \vspace{-0.25cm}
\begin{python}
def u0_boundary(x, on_boundary):
    return on_boundary
\end{python}

  \bigskip

  You may want to experiment with the definition of the boundary:

\begin{python}
def u0_boundary(x):
    return x[0] < DOLFIN_EPS or \
           x[1] > 1.0 - DOLFIN_EPS
\end{python}
\vspace{-0.5cm}
\begin{python}
def u0_boundary(x):
    return near(x[0], 0.0) or near(x[1], 1.0)
\end{python}
\vspace{-0.5cm}
\begin{python}
def u0_boundary(x, on_boundary):
    return on_boundary and x[0] > DOLFIN_EPS
\end{python}

\end{frame}
                  %% Alt 1
%\begin{frame}[fragile]
  \frametitle{Step by step: defining a boundary condition}

  The following code defines a Dirichlet boundary condition:

\begin{python}
bc = DirichletBC(V, u0, u0_boundary)
\end{python}

  \bigskip

  This boundary condition states that a function in the function space
  defined by \emp{V} should be equal to \emp{u0} on the boundary
  defined by \emp{u0\_boundary}

  \bigskip

  Note that the above line does not yet apply the boundary condition
  to all functions in the function space

\end{frame}
        %% Alt 1
\begin{frame}[fragile]
  \frametitle{Step by step: defining a boundary condition}

  The following code defines a Dirichlet boundary condition:
\vspace{-1em}
\begin{python}
bc = DirichletBC(V, u0, "on_boundary")
\end{python}

  This boundary condition states that a function in the function space
  defined by \emp{V} should be equal to \emp{u0} on the domain defined
  by \emp{"on\_boundary"}

  \bigskip

  Note that the above line does not yet apply the boundary condition
  to all functions in the function space

\end{frame}
 %% Alt 2
\begin{frame}[fragile, shrink=5]
  \frametitle{Step by step: more about defining domains}
  For a Dirichlet boundary condition, a simple domain can be defined
  by a string
  \vspace{-1em}
  \begin{python}
"on_boundary" # The entire boundary
  \end{python}

  Alternatively, domains can be defined by subclassing \emp{SubDomain}
  \vspace{-1em}
  \begin{python}
class Boundary(SubDomain):
    def inside(self, x, on_boundary):
        return on_boundary
  \end{python}

  You may want to experiment with the definition of the boundary:
  \vspace{-1em}
  \begin{python}
"near(x[0], 0.0)" # x_0 = 0
"near(x[0], 0.0) || near(x[1], 1.0)"
  \end{python}

  There are many more possibilities, see
  \vspace{-1em}
  \begin{python}
help(SubDomain)
help(DirichletBC)
  \end{python}

\end{frame}
      %% Alt 2

\begin{frame}[fragile]
  \frametitle{Step by step: defining the right-hand side}

  The right-hand side $f = - 6$ may be defined as follows:
  \begin{python}
f = Expression("-6.0", degree=0)
  \end{python}

  \bigskip

  or (more efficiently) as
  \begin{python}
f = Constant(-6.0)
  \end{python}

\end{frame}

\begin{frame}[fragile]
  \frametitle{Step by step: defining variational problems}

  Variational problems are defined in terms of \emph{trial} and
  \emph{test} functions:
  \begin{python}
u = TrialFunction(V)
v = TestFunction(V)
  \end{python}

  \bigskip

  We now have all the objects we need in order to specify the bilinear
  form $a(u,v)$ and the linear form $L(v)$:
  \begin{python}
a = inner(grad(u), grad(v))*dx
L = f*v*dx
  \end{python}

\end{frame}

\begin{frame}[fragile]
  \frametitle{Step by step: solving variational problems}

  Once a variational problem has been defined, it may be solved
  by calling the \emp{solve} function:

  \begin{python}
u = Function(V)
solve(a == L, u, bc)
  \end{python}

  \bigskip

  Note the reuse of the variable name \emp{u} as both a \emp{TrialFunction}
  in the variational problem and a \emp{Function} to store the solution.

\end{frame}

% \begin{frame}[fragile]
  \frametitle{Step by step: post-processing}

  The solution and the mesh may be plotted by simply calling:
  \vspace{-0.25cm}
  \begin{python}
plot(u)
plot(mesh)
interactive()
  \end{python}

  \bigskip

  The \emp{interactive()} call is necessary for the plot to remain on the
  screen and allows the plots to be rotated, translated and zoomed

  \bigskip

  For postprocessing in ParaView or MayaVi, store the solution in VTK
  format:
  \begin{python}
file = File("poisson.pvd")
file << u
  \end{python}

\end{frame}
 Uncomment this if you use VTK plotting
\begin{frame}[fragile]
  \frametitle{Step by step: post-processing using Notebooks}

  Add these incantations on top (after importing dolfin/fenics)
  \vspace{-1em}
  \begin{python}
import pylab
%matplotlib inline
parameters["plotting_backend"] = "matplotlib"
    \end{python}

  The solution and the mesh may be plotted by simply calling:
  \vspace{-1em}
  \begin{python}
plot(u)
pylab.show()
plot(mesh)
pylab.show()
  \end{python}

  For postprocessing in ParaView or MayaVi, store the solution in VTK
  format:
  \vspace{-1em}
  \begin{python}
file = File("poisson.pvd")
file << u
  \end{python}

\end{frame}


% Comment this next slide out if you are confident that the students
% know how to write and run a Python program, or if you are using
% e.g. Notebooks
% \begin{frame}[fragile]
\frametitle{Python/FEniCS programming 101}

\begin{enumerate}
\item
  Open a file with your favorite text editor (Emacs :-) ) and name the
  file something like \texttt{test.py}
  \bigskip
\item
  Write the following in the file and save it:
\vspace{-1em}
  \begin{python}
from fenics import *
print(dolfin.__version__)
  \end{python}
  \bigskip
\item
  Run the file/program by typing the following in a terminal (with
  FEniCS setup):
\vspace{-1em}
 \begin{python}
$ python test.py
 \end{python}
\end{enumerate}

\end{frame}


\begin{frame}
  \frametitle{\emph{The FEniCS challenge!}}

  Solve the partial differential equation
  \begin{equation*}
    -\Delta u = f
  \end{equation*}
  with homogeneous Dirichlet boundary conditions on the unit square
  for $f(x, y) = 2\pi^2\sin(\pi x)\sin(\pi y)$. Plot the error in the
  $L^2$ norm as function of the mesh size $h$ for a sequence of
  refined meshes. Try to determine the convergence rate.

  \begin{itemize}
  \item
    \emph{Who can obtain the smallest error?}
  \item
    \emph{Who can compute a solution with an error smaller than
    $\epsilon = 10^{-6}$ in the fastest time?}
  \end{itemize}

  \emph{The best students(s) will be rewarded with a FEniCS surprise!}

  \bigskip
  Hints: \emp{help(errornorm)}, \emp{help(assemble)}

\end{frame}


\end{document}
