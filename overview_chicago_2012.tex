\documentclass{fenicscourse}

\begin{document}

\fenicslecture{Overview}
              {Anders Logg \\
               Marie E. Rognes}

\begin{frame}
  \frametitle{Course outline}

  \begin{enumerate}
  \item[L0]
    Introduction to FEM (Fri Nov 5)
  \item[L1]
    Introduction to FEniCS (Tue Nov 9)
  \item[L2]
    Static linear PDEs (Tue Nov 9)
  \item[L3]
    \textcolor{grey}{Static nonlinear PDEs}
  \item[L4]
    \textcolor{grey}{Time-dependent PDEs}
  \item[L5]
    \textcolor{grey}{Mixed problems}
  \item[L6]
    Static hyperelasticity (Thur 11 Nov)
  \item[L7]
    \textcolor{grey}{Dynamic hyperelasticity}
  \item[L8]
    Incompressible Navier--Stokes (Tue 16 Nov)
  \end{enumerate}

  \bigskip

  \emph{There will be hands-on FEniCS challenges at the end of each
    lecture so be alert!}

  %% {\footnotesize Lectures can be downloaded from
  %%   \url{http://fenicsproject.org/course/}}

\end{frame}

\begin{frame}[fragile]
\frametitle{Some practicalities}

\begin{itemize}
\item Course material: \\
  \emp{www.fenicsproject.org/pub/course/}
\item Web resources: \\
  \emp{www.fenicsproject.org} \\
  \emp{www.fenicsproject.org/documentation}
\item
  The FEniCS Project is an umbrella for several software components:
  you will interact with the main user interface: DOLFIN.
\item
  In particular, we will use the Python interface to DOLFIN (there is
  also a C++ interface)
\item
  Write your program (\emp{foo.py}) in your favorite text editor, save
  it and run
  \vspace{-1em}
  \begin{bash}
python foo.py
  \end{bash}
\end{itemize}

\end{frame}

\begin{frame}[fragile]
\frametitle{First test}

Check that the DOLFIN installation operates normally and that you have
the right version (DOLFIN 1.0). Open a Python shell by writing (in a
terminal)
\vspace{-1em}
\begin{bash}
  ipython
\end{bash}
and then do
\vspace{-1em}
\begin{python}
import dolfin
print dolfin.__version__
\end{python}
Expected output: '1.0.0'

\end{frame}

\end{document}
