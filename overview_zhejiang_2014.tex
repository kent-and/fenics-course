\documentclass{fenicscourse}

\begin{document}

\fenicslectureoverview{Simon Funke\vspace{0.25cm}}{Zhejiang University\\ Mar 10--14 2014}

\begin{frame}
  \frametitle{Course outline}

  % Note to lecturers: use \textcolor{grey}{Lecture title} to mark
  % lectures that are not included in a course
  % Basic structure:


  % ** Monday - Tuesday ** 
  % Introduction to solving PDEs (Monday, Tuesday)
  % 
  % ** Wednesday - Thursday ** 
  % Computing sensitivities. 
  %      Part 1: Adjoint and TLM solutions, steady state example, Taylor test. 
  %      Part 2: Introduction to dolfin-adjoint, annotation, computing gradients and Hessians
  %
  % From sensitivities to optimisation
  %      Part 1: Optimisation algorithms: Steepest descent, Newton, quasi-Newton
  %      Part 2: Bound constraints, (in-)equality constraints for the control
  %
  % ** Friday **
  %   Part 1: Optimality conditions, KKT approach
  %   Part 2: Coding Session, Optimal control of the Navier-Stokes equation, Topology optimisation

  \footnotesize

  \hspace{2cm}
  \begin{enumerate}
    \hrule
  \item[\textcolor{black}{\it Mon} L00]
    Introduction to FEM
  \item[L01]
    Introduction to FEniCS
  \item[L02]
    Static linear PDEs
    \\[0.5em]\hrule
  \item[\textcolor{black}{\it Tue} L03]
    Static nonlinear PDEs
  \item[L04]
    Time-dependent PDEs
    \\[0.5em]\hrule
  \item[\textcolor{black}{\it Wed} L05]
    Happy hacking: Tools, tips and coding practices
  \item[L06]
    Computing sensitivites 
    \\[0.5em]\hrule
  \item[\textcolor{black}{\it Thu} L07]
    Introduction to dolfin-adjoint
  \item[L08]
    From sensitivites to optimisation
    \\[0.5em]\hrule
  \item[\textcolor{black}{\it Fri} L09]
    One-shot optimisation
  \item[L10]
    Coding session
    \\[0.5em]\hrule
  \end{enumerate}

  \normalsize

  {\footnotesize Lectures can be downloaded from
    \url{http://fenicsproject.org/pub/course/}}
\end{frame}

% The quickest FEniCS intro
\begin{frame}
\medskip
\includegraphics[width=0.99\textwidth]{png/fenics_banner.png}
\begin{columns}[c]
\begin{column}{0.4\textwidth}
\bf{The FEniCS Project is a collection of open-source software
  components aimed at the numerical solution of partial differential
  equations using finite element methods}
\end{column}
\begin{column}{0.7\textwidth}
  \begin{block}{Key distinguishing features}
  \begin{itemize}
  \item
    FEniCS (Python/C++) code is quick to write and easy to read
  \item
    `Any' finite element formulation of 'any' partial differential
    equation can be coded
  \item
    Automated code generation is used to create 
    create efficient, low-level code
  \item
    Performance -- implicit problems with millions DOFs
    can be solved in minutes
  \item
    Automatic adjoint derivation for sensitivity computation 
    and PDE-constrained optimisation
  \end{itemize}
  \end{block}
\end{column}
\end{columns}
  \begin{center}
    \colemph{\url{http://fenicsproject.org/}}
  \end{center}
\end{frame}

\begin{frame}
\frametitle{FEniCS has been used for a wide range of
  equations and applications}

{\tiny Reaction-diffusion equations; Stokes with or without nonlinear
  viscosity; compressible and incompressible Navier--Stokes; RANS
  turbulence models; shallow water equations; Bidomain equations;
  nonlinear and linear elasticity; nonlinear and linear
  viscoelasticity; Schr\"odinger; Biot's equations for porous media,
  fracture mechanics, electromagnetism, liquid crystals including
  liquid crystal elastomers, combustion, ... and coupled systems of
  the above, ...}

\begin{center}
\includegraphics[width=0.24\textwidth]{png/g_el_plusx.png}
\includegraphics[width=0.24\textwidth]{png/unhealthy_v_at_T200.png}
\includegraphics[width=0.24\textwidth]{png/circle_of_willis_simulation.png}
\end{center}

{\tiny for simulating blood flow, computing calcium release in cardic
  tissue, computing the cardiac potential in the heart, simulating
  mantle convection, simulating melting ice sheets, computing the
  optimal placement of tidal turbines, simulating and reconstructing
  tsunamis, simulating the flow of cerebrospinal fluid and the
  deformation of the spinal cord, simulating waveguides, ... }

\end{frame}


% Learning more than what this course covers
\begin{frame}
    \frametitle{Sounds great, but how do I find my way through the
    jungle?}
    \begin{center}
        \includegraphics[width=0.8\textwidth]{jpg/jungle10.jpg}
    \end{center}
\end{frame}

\begin{frame}
    \frametitle{Three survival advices}
    \begin{columns}[c]
        \begin{column}{0.33\textwidth}
            \begin{center}
                \includegraphics[width=0.99\textwidth]{png/python_logo.png}
            \end{center}
        \end{column}
        \begin{column}{0.33\textwidth}
            \begin{center}
                \includegraphics[width=0.99\textwidth]{jpg/documentation.jpg}\\
            \end{center}
        \end{column}
        \begin{column}{0.33\textwidth}
            \begin{center}
                \includegraphics[width=0.99\textwidth]{jpg/question-blue.jpg}
            \end{center}
        \end{column}
    \end{columns}
    \begin{columns}[t]
        \begin{column}{0.33\textwidth}
            \begin{center}
                \colemph{Use the right Python tools}
            \end{center}
        \end{column}
        \begin{column}{0.33\textwidth}
            \begin{center}
                \colemph{Explore the documentation}
            \end{center}
        \end{column}
        \begin{column}{0.33\textwidth}
            \begin{center}
                \colemph{Ask, report and request}
            \end{center}
        \end{column}
    \end{columns}
\end{frame}


% TODO: Update webpage images when readthedocs work is completed

% Demos on old page
\begin{frame}
  \begin{center}
     {\includegraphics[width=0.80\textwidth]{png/fenics-doc-webpage-5.png}}
    \small
    \colemph{\url{http://fenicsproject.org/documentation/}}
  \end{center}
\end{frame}

% Reference docs on old page
\begin{frame}
  \begin{center}
     {\includegraphics[width=0.80\textwidth]{png/fenics-doc-webpage-6.png}}
    \small
    \colemph{\url{http://fenicsproject.org/documentation/}}
  \end{center}
\end{frame}

% Currently migrating
\begin{frame}
  \begin{center}
     {\includegraphics[width=0.80\textwidth]{png/fenics-readthedocs-webpage-1.png}}
    \small
    \colemph{\url{http://fenics.readthedocs.org/}}
  \end{center}
\end{frame}

%\begin{frame}
    \frametitle{Development community is organized via bitbucket.org}
    \begin{center}
        \includegraphics[height=0.75\textheight]{png/fenics-bitbucket-webpage.png}
        \vspace{1em}
        \small
        \colemph{\url{http://bitbucket.org/fenics-project/}}
    \end{center}
\end{frame}
\begin{frame}
    \frametitle{Community help is available via QA forum}
    \begin{center}
        \includegraphics[width=1.0\textwidth,height=0.7\textheight]{png/fenics-qa-website.png}
        \vspace{1em}
        \small
        \colemph{\url{https://fenicsproject.org/qa}}
    \end{center}
\end{frame}

\begin{frame}
    \frametitle{Getting help from the FEniCS community}
    \begin{center}
        \includegraphics[height=0.75\textheight]{png/qa_overview.png}
        \vspace{1em}
        \small
        \colemph{\url{http://fenicsproject.org/qa/}}
    \end{center}
\end{frame}

\begin{frame}
    \frametitle{Getting help from the FEniCS community}
    \begin{center}
        \includegraphics[height=0.75\textheight]{png/qa_example.png}
        \vspace{1em}
        \small
        \colemph{\url{http://fenicsproject.org/qa/}}
    \end{center}
\end{frame}

\begin{frame}
  \frametitle{Community resources}

  \linespread{1.5}
  \begin{itemize}
  \item
    The FEniCS mailing list

    \url{fenics@fenicsproject.org}
  \item
    The FEniCS QA forum

    \url{http://fenicsproject.org/qa/}
  \item
    The FEniCS Google+ community

    \url{http://plus.google.com/}
  \item
    Twitter

    \url{\#fenicsproject}
  \item
    The FEniCS developer site (Bitbucket)

    \url{https://bitbucket.org/fenics-project/}
  \end{itemize}
  \linespread{1.0}

  \vspace{0.1cm}

  \begin{center}
    \underline{\colemph{\url{http://fenicsproject.org/}}}
  \end{center}

\end{frame}


\begin{frame}
  \frametitle{Installation alternatives}

  % FEniCS uses standard setup.py and cmake tools
  % but dependencies are tricky to configure.

  \begin{tabular}{cp{10cm}}
    \includegraphics[height=1cm]{png/docker_logo.png} &
    \begin{minipage}{10cm}
      \ding{43} Docker images on Linux, Mac, Windows
      \vspace{0.6cm}
    \end{minipage}
    \\
    \includegraphics[height=1cm]{png/source.png} &
    \begin{minipage}{10cm}
      \ding{43} Build from source with Hashdist (fenics-install.sh)
      \vspace{0.8cm}
    \end{minipage}
    \\
    \includegraphics[height=1cm]{png/ubuntu_logo.png} &
    \begin{minipage}{10cm}
      \ding{43} PPA with apt packages for Debian and Ubuntu
      \vspace{0.6cm}
    \end{minipage}
    \\
    \includegraphics[height=1cm]{png/mac_osx_logo.png} &
    \begin{minipage}{10cm}
      \ding{43} Drag and drop installation on Mac OS X
      \vspace{0.8cm}
    \end{minipage}
  \end{tabular}

  \begin{center}
    \colemph{\url{http://fenicsproject.org/download/}}
  \end{center}

\end{frame}


\begin{frame}{Let's get started and remember:}

\linespread{2.0}
\bigskip
\begin{itemize}
\item
{\footnotesize \textbf{Lectures} can be downloaded from
  \url{http://fenicsproject.org/pub/course/lectures/}}

\item
{\footnotesize \textbf{Data} for exercises can be downloaded from
  \url{http://fenicsproject.org/pub/course/data/}}

\item
{\footnotesize \textbf{Solutions} for exercises can be downloaded from
  \url{http://fenicsproject.org/pub/course/src/} \\
(Secret password needed!)
}
\end{itemize}
\linespread{1.0}

\end{frame}

\end{document}
