\documentclass{fenicscourse}

\begin{document}

%\fenicslecture{Overview}{Martin Sandve Aln{\ae}s}
\fenicslectureoverview{Martin Sandve Aln{\ae}s}{Lund, June 9-10 2015}

\begin{frame}
  \frametitle{Course outline}

  %
  % Breaks:
  % Tuesday 9-16 (optional tutoring until 17)
  % 1015-1030 Coffee
  % 1200-1300 Lunch
  % 1445-1500 Coffee
  %
  % Wednesday 9-1430 (+ questions at the end until I have to run)
  % 1015-1030 Coffee
  % 1200-1300 Lunch
  %

  %
  % Tuesday
  % *** 9-12
  % 0900-0930   This overview                    30min
  % 0930-1015   L02 Static linear PDEs           45 min lecture
  % 1015-1030   Coffee
  % 1030-1115   L02 Static linear PDEs           45 min challenge
  % 1115-1200   L03 static nonlinear pdes        45 min lecture only
  %
  % *** 12-13 Lunch
  %
  % *** 13-17
  % 1300-1445
  % 1300-1430  L04 Time-dependent PDEs          45 min lecture + 45 min challenge
  % 1430-1530  L05 Happy hacking                30 min lecture + 30 min challenge
  %
  % *** 1500-1700
  % 1500-1700
  %   L06 Static hyperelasticity                30 min lecture + 30-90 min challenge
  %
  % Wednesday
  % *** 9-12
  %   L07 Dynamic hyperelasticity      30 min lecture + 45 min hands-on
  %   Coffee 1015-1030
  %   L08 The Stokes problem           45 min (no challenge)
  %   L09 Incompressible Navier-Stokes 45 min lecture
  %
  % *** 12-13 Lunch
  %
  % *** 13-1430
  %   L09 Incompressible Navier-Stokes 45 min hands-on
  %   Encourage exploration
  %

  % Note to lecturers: use \textcolor{grey}{Lecture title} to mark
  % lectures that are not included in a course

  \small

  \begin{enumerate}
  \item[L00]
    \textcolor{grey}{Introduction to FEM}
  \item[L01]
    Introduction to FEniCS
    % TODO: Create new community webpage snapshots for bitbucket, remove launchpad.
  \item[L02]
    Static linear PDEs
  \item[L03]
    Static nonlinear PDEs
  \item[L04]
    Time-dependent PDEs
  \item[L05]
    Happy hacking: Tools, tips and coding practices
  \item[L06]
    Static hyperelasticity
  \item[L07]
    Dynamic hyperelasticity
  \item[L08]
    The Stokes problem
  \item[L09]
    Incompressible Navier--Stokes
  \item[L10]
    \textcolor{grey}{Discontinuous Galerkin methods for elliptic equations}
  \item[L11]
    \textcolor{grey}{A posteriori error estimates and adaptivity}
  \end{enumerate}

  \normalsize

  {\footnotesize Lectures can be downloaded from
    \url{http://fenicsproject.org/pub/course/}}

\end{frame}


% The quickest FEniCS intro
\begin{frame}
\medskip
\includegraphics[width=0.99\textwidth]{png/fenics_banner.png}
\begin{columns}[c]
\begin{column}{0.4\textwidth}
\bf{The FEniCS Project is a collection of open-source software
  components aimed at the numerical solution of partial differential
  equations using finite element methods}
\end{column}
\begin{column}{0.7\textwidth}
  \begin{block}{Key distinguishing features}
  \begin{itemize}
  \item
    FEniCS (Python/C++) code is quick to write and easy to read
  \item
    `Any' finite element formulation of 'any' partial differential
    equation can be coded
  \item
    Automated code generation is heavily used under the hood to
    create efficient, specialized, low-level code
  \item
    Performance -- implicit problems with over $12\, 000\, 000\, 000$
    degrees of freedom can be solved in a couple of minutes
  \end{itemize}
  \end{block}
\end{column}
\end{columns}
  \begin{center}
    \colemph{\url{http://fenicsproject.org/}}
  \end{center}
\end{frame}

\begin{frame}
\frametitle{FEniCS has been used for a wide range of
  equations and applications}

{\tiny Reaction-diffusion equations; Stokes with or without nonlinear
  viscosity; compressible and incompressible Navier--Stokes; RANS
  turbulence models; shallow water equations; Bidomain equations;
  nonlinear and linear elasticity; nonlinear and linear
  viscoelasticity; Schr\"odinger; Biot's equations for porous media,
  fracture mechanics, electromagnetism, liquid crystals including
  liquid crystal elastomers, combustion, ... and coupled systems of
  the above, ...}

\begin{center}
\includegraphics[width=0.24\textwidth]{png/g_el_plusx.png}
\includegraphics[width=0.24\textwidth]{png/unhealthy_v_at_T200.png}
\includegraphics[width=0.24\textwidth]{png/circle_of_willis_simulation.png}
\end{center}

{\tiny for simulating blood flow, computing calcium release in cardic
  tissue, computing the cardiac potential in the heart, simulating
  mantle convection, simulating melting ice sheets, computing the
  optimal placement of tidal turbines, simulating and reconstructing
  tsunamis, simulating the flow of cerebrospinal fluid and the
  deformation of the spinal cord, simulating waveguides, ... }

\end{frame}


% Show a simple taster
\begin{frame}
  \frametitle{Hello World in FEniCS: problem formulation}

  \begin{block}{Poisson's equation}
    \vspace{-0.5cm}
    \begin{displaymath}
      \begin{split}
        -\Delta u &= f \quad \mbox{ in } \Omega \\
        u &= 0 \quad \mbox { on } \partial\Omega
      \end{split}
    \end{displaymath}
  \end{block}

  \begin{block}{Finite element formulation}
    \vspace{1ex}
    Find $u \in V$ such that
    \begin{displaymath}
      \underbrace{\int_{\Omega} \nabla u \cdot \nabla v \dx}_{\textcolor{fenicsred}{a(u,v)}}
      = \underbrace{\int_{\Omega} f \, v \dx}_{\textcolor{fenicsred}{L(v)}}
      \quad \foralls v \in V
    \end{displaymath}
  \end{block}

\end{frame}

\begin{frame}[fragile]
  \frametitle{Hello World in FEniCS: implementation}

    \begin{python}
from fenics import *

mesh = UnitSquareMesh(32, 32)

V = FunctionSpace(mesh, "Lagrange", 1)
u = TrialFunction(V)
v = TestFunction(V)
f = Expression("x[0]*x[1]", degree=2)

a = dot(grad(u), grad(v))*dx
L = f*v*dx

bc = DirichletBC(V, 0.0, DomainBoundary())

u = Function(V)
solve(a == L, u, bc)
plot(u)
    \end{python}

\end{frame}

\begin{frame}
  \frametitle{Basic API}

  \begin{itemize}
  \item
    \texttt{Mesh},
    \texttt{Vertex},
    \texttt{Edge},
    \texttt{Face},
    \texttt{Facet},
    \texttt{Cell}
  \item
    \texttt{FiniteElement}, \texttt{FunctionSpace}
  \item
    \texttt{TrialFunction},
    \texttt{TestFunction},
    \texttt{Function}
  \item
    \texttt{grad()}, \texttt{curl()}, \texttt{div()}, \ldots
  \item
    \texttt{Matrix}, \texttt{Vector}, \texttt{KrylovSolver}, \texttt{LUSolver}
  \item
    \texttt{assemble()}, \texttt{solve()}, \texttt{plot()}
  \end{itemize}

  \vspace{1cm}

  \begin{itemize}
  \item
    Python interface generated semi-automatically by SWIG
  \item
    C++ and Python interfaces almost identical
  \end{itemize}

\end{frame}


% Learning more than what this course covers
\begin{frame}
    \frametitle{Sounds great, but how do I find my way through the
    jungle?}
    \begin{center}
        \includegraphics[width=0.8\textwidth]{jpg/jungle10.jpg}
    \end{center}
\end{frame}

\begin{frame}
    \frametitle{Three survival advices}
    \begin{columns}[c]
        \begin{column}{0.33\textwidth}
            \begin{center}
                \includegraphics[width=0.99\textwidth]{png/python_logo.png}
            \end{center}
        \end{column}
        \begin{column}{0.33\textwidth}
            \begin{center}
                \includegraphics[width=0.99\textwidth]{jpg/documentation.jpg}\\
            \end{center}
        \end{column}
        \begin{column}{0.33\textwidth}
            \begin{center}
                \includegraphics[width=0.99\textwidth]{jpg/question-blue.jpg}
            \end{center}
        \end{column}
    \end{columns}
    \begin{columns}[t]
        \begin{column}{0.33\textwidth}
            \begin{center}
                \colemph{Use the right Python tools}
            \end{center}
        \end{column}
        \begin{column}{0.33\textwidth}
            \begin{center}
                \colemph{Explore the documentation}
            \end{center}
        \end{column}
        \begin{column}{0.33\textwidth}
            \begin{center}
                \colemph{Ask, report and request}
            \end{center}
        \end{column}
    \end{columns}
\end{frame}


% TODO: Update webpage images when readthedocs work is completed

% Demos on old page
\begin{frame}
  \begin{center}
     {\includegraphics[width=0.80\textwidth]{png/fenics-doc-webpage-5.png}}
    \small
    \colemph{\url{http://fenicsproject.org/documentation/}}
  \end{center}
\end{frame}

% Reference docs on old page
\begin{frame}
  \begin{center}
     {\includegraphics[width=0.80\textwidth]{png/fenics-doc-webpage-6.png}}
    \small
    \colemph{\url{http://fenicsproject.org/documentation/}}
  \end{center}
\end{frame}

% Currently migrating
\begin{frame}
  \begin{center}
     {\includegraphics[width=0.80\textwidth]{png/fenics-readthedocs-webpage-1.png}}
    \small
    \colemph{\url{http://fenics.readthedocs.org/}}
  \end{center}
\end{frame}

\begin{frame}
    \frametitle{Development community is organized via bitbucket.org}
    \begin{center}
        \includegraphics[height=0.75\textheight]{png/fenics-bitbucket-webpage.png}
        \vspace{1em}
        \small
        \colemph{\url{http://bitbucket.org/fenics-project/}}
    \end{center}
\end{frame}
\begin{frame}
    \frametitle{Community help is available via QA forum}
    \begin{center}
        \includegraphics[width=1.0\textwidth,height=0.7\textheight]{png/fenics-qa-website.png}
        \vspace{1em}
        \small
        \colemph{\url{https://fenicsproject.org/qa}}
    \end{center}
\end{frame}


\begin{frame}
  \frametitle{Installation alternatives}

  % FEniCS uses standard setup.py and cmake tools
  % but dependencies are tricky to configure.

  \begin{tabular}{cp{10cm}}
    \includegraphics[height=1cm]{png/docker_logo.png} &
    \begin{minipage}{10cm}
      \ding{43} Docker images on Linux, Mac, Windows
      \vspace{0.6cm}
    \end{minipage}
    \\
    \includegraphics[height=1cm]{png/source.png} &
    \begin{minipage}{10cm}
      \ding{43} Build from source with Hashdist (fenics-install.sh)
      \vspace{0.8cm}
    \end{minipage}
    \\
    \includegraphics[height=1cm]{png/ubuntu_logo.png} &
    \begin{minipage}{10cm}
      \ding{43} PPA with apt packages for Debian and Ubuntu
      \vspace{0.6cm}
    \end{minipage}
    \\
    \includegraphics[height=1cm]{png/mac_osx_logo.png} &
    \begin{minipage}{10cm}
      \ding{43} Drag and drop installation on Mac OS X
      \vspace{0.8cm}
    \end{minipage}
  \end{tabular}

  \begin{center}
    \colemph{\url{http://fenicsproject.org/download/}}
  \end{center}

\end{frame}



\begin{frame}{In this course we'll be running on the alarik HPC cluster}
\begin{itemize}
\item \textbf{Log in to alarik:} \texttt{ssh alarik.lunarc.lu.se}
\item \textbf{Change directory:} \texttt{cd \$\{SNIC\_NOBACKUP\}}
\item \textbf{Copy example jobscript:}\\
\texttt{cp -r /alarik/nobackup/i\_p/martinal/pub .}
\item \textbf{Try scheduling a job:}\\
\texttt{cd pub/job}\\
\texttt{sbatch jobscript.sh}\\
\texttt{cat slurm-somejobid.out}\\
\emph{Successfully imported dolfin version 1.5.0}
\end{itemize}
\end{frame}


\begin{frame}{Let's get started and remember:}

\linespread{2.0}
\bigskip
\begin{itemize}
\item
{\footnotesize \textbf{Lectures} can be downloaded from
  \url{http://fenicsproject.org/pub/course/lectures}}

\item
{\footnotesize \textbf{Data} for exercises can be downloaded from
  \url{http://fenicsproject.org/pub/course/data} \\
(Or copy from the .../pub/data directory on alarik)}

\item
{\footnotesize \textbf{Solutions} for exercises can be downloaded from
  \url{http://fenicsproject.org/pub/course/src} \\
(Secret password needed!)
}
\end{itemize}
\linespread{1.0}\

\end{frame}

\end{document}
